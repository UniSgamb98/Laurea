\chapter*{Sommario}
\label{sommario}

\addcontentsline{toc}{chapter}{Sommario}

In tempi con così tanta tecnologia, sembra che i giovani siano sempre più propensi a distrarsi. Questa tendenza è molto probabilmente provocata dalla diffusione di \textit{social media} come \textit{Instagram, Tik Tok, Facebook} e molti altri. Questi infatti condividono tutti la stessa modalità di trasmissione di contenuti: brevi video carichi d'informazioni che ritraggono spesso viste mozzafiato, luoghi paradisiaci, imprese incredibili o più semplicemente video montaggi di scene divertenti oppure stupide.
Questo problema di mantenimento di attenzione è già stato incontrato e risolto circa vent'anni fa dagli sviluppatori di videogiochi che cercando un modo per riuscire a mantenere l'interesse dell'utente alto, con lo scopo che esso continui a fare d'uso della propria applicazione, realizzarono che introducento determinati elementi nei propri giochi avevano più probabilità di successo nel farlo. Punti, badge, classifiche, missioni sono solo alcuni degli aspetti che rendono i giochi e soprattutto i videogiochi così entusiasmanti da assorbirci delle ore senza annoiarci.
Nel 1980 Thomas Malone, in un articolo che venne pubblicato al MIT[1], condusse uno studio cercando quali di queste caratteristiche sono la causa del coinvolgimento dell'utente e come queste riescono ad essere così efficaci nel loro compito. In queste condizioni, è nata così la tecnica conosciuta con il nome di \textit{Gamification}. Essa consiste nel prendere gli elementi che rendono divertenti i videogiochi e includerli in un contesto educativo. Questo nuovo modo di insegnare si basa sul concetto di sfruttare la predisposizione psicologica a impegnarsi nel gioco come potenziale mezzo per rendere più coinvolgenti le attività del mondo reale. Nella sua essenza rende l'attività di apprendimento come un gioco.
\\
\\
Della \textit{Gamification} come medoto di insegnamento ne è già stata appurata l'efficacia. Pertanto, in questa tesi andremo ad affrontare lo stesso tema ma in un contesto lavorativo. Per prima cosa studiaremo in che modo questo rivoluzionario sistema funzioni, quali aspetti dell'insegnamento soddisfa e che livello di comprensione ci possiamo aspettare. Analizzeremo qualche esempio in vari campi di studio e ne confronteremo i risultati. Per fare tutto ciò utilizzeremo andremo a utilizzare il concetto della tassonomia di Bloom per sezionare il problema e capire come funziona l'apprendimento in generale e adoperando il metodo del cono di Edgar Dale come raggiungere la migliore compresione possibile del tema nel modo più veloce ed efficace.
Quindi in modo analogo andremo a fare lo stesso studio in ambito lavorativo per confrontarne le similitudini e capire come si possa tradurre la \textit{Gamification} in una azienda. Esamineremo diversi ruoli, da mansioni esecutive a posizioni più importanti come quelle di amministrazione.
\\
\\
\\
\\
\\
\\
\\
\\
\\
\\
