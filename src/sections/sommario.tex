\chapter*{Sommario}
\label{sommario}

\addcontentsline{toc}{chapter}{Sommario}

In tempi con così tanta tecnologia, sembra che i giovani siano sempre più proni a distrarsi. Molto probabilmente provacata dai \textit{social media} come \textit{Instagram, Tik Tok, Facebook} e molti altri. Questi infatti condividono tutti la stessa modalità di diffusione di informazioni. Tramite brevi video carichi d'informazioni che ritraggono spesso viste mozzafiato, luoghi paradisiaci, imprese incredibili o più semplicemente video montaggi di scene divertenti oppure stupide.
Questo problema di mantenimento di attenzione fu già incontrato circa vent'anni fa dagli sviluppatori di videogiochi che cercando un modo per riuscire a detenere l'interesse dell'utente alto, in modo che esso continui a fare d'uso dell'applicazione, realizzarono che introducento determinati elementi nei propri giochi avevano più successo nel farlo. Punti, badge, classifiche, missioni sono solo alcuni di questi elementi che rendono i giochi e soprattutto i videogiochi così entusiasmanti da assorbirci delle ore senza annoiarci.
Tali aspetti che rendono i videogiochi così coinvolgenti furono studiati per riuscire a comprendere meglio perchè fosse questo il caso, in un articolo pubblicato nel 1980 da \textit{Thomas Malone} al MIT [1]. In queste condizioni è nata la tecnica così conosciuta con il nome di \textit{Gamification}. La \textit{Gamification} prende gli elementi che rendono divertenti i videogiochi e li include in un contesto educativo.
Questa tecnica si basa sul concetto di sfruttare la predisposizione psicologica a impegnarsi nel gioco come potenziale mezzo per rendere più coinvolgenti le attività del mondo reale. Nella sua essenza rende gioco un'attività come l'apprendimento.

Della \textit{Gamification} come medologia di insegnamento ne è già stata appurata l'efficacia. Pertanto, in questa tesi andremo ad analizzarne l'efficacia in un contesto lavorativo. Andremo per prima cosa a studiare in che modo questa rivoluzionaria tecnica funziona, quali aspetti dell'insegnamento soddisfa e che livello di comprensione ci possiamo aspettare. Analizzeremo qualche esempio in vari campi di studio e ne confronteremo i risultati. Per fare tutto ciò utilizzeremo andremo a utilizzare il concetto della tassonomia di Bloom per sezionare il problema e capire come funziona l'apprendimento in generale e adoperando il metodo del cono di Edgar Dale come raggiungere la migliore compresione possibile del tema nel modo più veloce ed efficace.
Quindi in modo analogo andremo a fare lo stesso studio in ambito lavorativo per confrontarne le similitudini e capire come si possa tradurre la \textit{Gamification} in una azienda. Esamineremo diversi ruoli, da mansioni esecutive a posizioni più importanti come quelle di amministrazione.



  Sommario è un breve riassunto del lavoro svolto dove si descrive l'obiettivo, l'oggetto della tesi, le
metodologie e le tecniche usate, i dati elaborati e la spiegazione delle conclusioni alle quali siete arrivati.

Il sommario dell'elaborato consiste al massimo di 3 pagine e deve contenere le seguenti informazioni:
\begin{itemize}
  \item contesto e motivazioni
  \item breve riassunto del problema affrontato
  \item tecniche utilizzate e/o sviluppate
  \item risultati raggiunti, sottolineando il contributo personale del laureando/a
\end{itemize}
