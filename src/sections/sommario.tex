\chapter*{Sommario}
\label{sommario}

\addcontentsline{toc}{chapter}{Sommario}

In tempi con così tanta tecnologia, sembra che i giovani siano sempre più propensi a distrarsi. Nonostante ciò piattaforme come \textit{Instagram, Tik Tok, Facebook} e molti altri riescono a catturare l'attenzione dei giovani per ore. Questi condividono tutti la stessa modalità di trasmissione di contenuti: brevi video carichi d'informazioni che ritraggono spesso viste mozzafiato, luoghi paradisiaci, imprese incredibili o più semplicemente video montaggi di scene divertenti oppure stupide.
Questo problema di mantenimento di attenzione è fu incontrato per la prima volta e risolto intorno agli anni '80 dagli sviluppatori di videogiochi che cercavano modi per riuscire a mantenere l'interesse dell'utente alto, con lo scopo che esso continui a fare d'uso della propria applicazione e realizzarono che introducento determinati elementi nei propri giochi avevano più probabilità di successo nel farlo. Punti, badge, classifiche, missioni sono solo alcuni degli aspetti che rendono i giochi e soprattutto i videogiochi così entusiasmanti da assorbirci delle ore senza annoiarci.
Thomas Malone, in un articolo che venne pubblicato dal MIT, condusse uno studio cercando quali di queste caratteristiche sono la causa del coinvolgimento dell'utente e come queste riescono ad essere così efficaci nel loro compito. In queste condizioni, è nata così la tecnica conosciuta con il nome di \textit{Gamification}. Essa consiste nel prendere gli elementi che rendono divertenti i videogiochi e includerli in un contesto educativo. Questo nuovo modo di insegnare si basa sul concetto di sfruttare la predisposizione psicologica a impegnarsi nel gioco come potenziale mezzo per rendere più coinvolgenti le attività del mondo reale. Nella sua essenza rende l'attività di apprendimento come un gioco.\\
\\
In questa tesi andremo a scoprire come realizzare un piano di studio e renderlo gamificato per riuscire a massimizzare l'efficacia degli insegnamenti che si vorrano condurre.\\
Per prima cosa studieremo come concepire obbiettivi di insegnamento, quali aspetti dell'insegnamento soddisfano e che livello di comprensione ci possiamo aspettare. Analizzeremo qualche esempio in vari campi della didattica e ne confronteremo i risultati. Per fare ciò ci avvarremo del concetto della tassonomia di Bloom, utile a esaminare i problemi di fronte a cui è posto un letto o utente, che ci consentirà di capire come funziona l'apprendimento in generale; adoperando il metodo del cono di Edgar Dale sarà illustrato come è viene raggiunta la migliore comprensione possibile di un determinato tema nel modo più veloce ed efficace. Quindi andremo a studiare quali sono gli elementi che rendono una piattoforma gamificata e come usarli. Infine andremo applicare tutto questo in ambito lavorativo, in una esperienza di lavoro che ho avuto piacere di affrontare immersa in un ambiente a contatto con le persone. 
\\
\\
\\
\\
\\
\\
\\
\\
\\
\\
