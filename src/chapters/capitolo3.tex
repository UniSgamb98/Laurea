\chapter{\textit{Gamificazione} all'opera}
\section{Parkour}
Una delle mie fin dai tempi delle superiori è il praticare lo sport del parkour e dopo 10 anni di pratica con la A.S.D. "Move Out Experience" sono diventato un loro istruttore. Con i suoi corsi di parkour "Move Out" insegna a bambini e ragazzi come praticare in sicurezza lo sport e istrurli sulla storia del parkour, come è nato, il suo sviluppo e la mentalità da assumere quando si pratica questa disciplina.

\subsection{Obbiettivi}
Ritornando a guardare il modello della piramide di Bloom cerchiamo di individuare una lista di obbiettivi che devono essere raggiunti lungo il periodo di un anno scolastico per insegnare il parkour.\\
\\
Per un \textit{"Traceur"} alle prime armi un buon inizio nella disciplina del parkour è riuscire ad imparare: il "step vault", il "monkey" e la "roll".\\
Il "step vault" abbreviato in "step" e il "monkey" o "saut de cat" sono due delle tecniche fondamentali nella disciplina del parkour che permettono di affrontare la maggior parte degli ostacoli che si potrebbero trovare lungo un percoso. Mentre la "roll" è un atterraggio di sicurezza che permette all'atleta, quando eseguita correttamente, di limitare o evitare possibili infortuni di fronte a salti particolarmente alti o quando qualcosa è andato storto nell'esecuzione di un salto.

Il modello di bloom ha 6 livelli: conoscenza, comprensione, applicazione, analisi, sintesi, valutazione.\\
Per il livello di conoscenza e compresione occorre conoscere il nome dei tre scavalcamenti e il movimento da compiere per eseguirli. In particolare per il monkey vault occorre prendere una breve rincorsa e saltare imitando un tuffo di testa e saltando a piedi sfalsati mentre si oscillano le braccia per creare ancora più spinta. Una volta spiccato il salto bisogna usare l'ostacolo che si vuole scavalcare per darsi ancora più spinta e orientarsi con i piedi in avanti per prepararsi all'atterraggio.\\
Per eseguire uno step vault invece si corre verso l'ostacolo e lo si scavalca appogiandoci sopra una mano e un piede alterni e facendo passare la gamba opposta al piede che poggia sull'ostacolo tra esso e il braccio utilizzato per il salto.\\
La roll invece è un atterraggio di sicurezza. quindi una volta atterrato con i piedi per terra ci si accovaccia assecondando il movimento della caduta, si poggiano le mani per terra e si rotola per terra facendo toccare per prima una delle due scapole e continuando la rotolata lungo una linea diagonale sulla schiena che parte da una scapola fino alla anca opposta.\\
\\
Per il livello di applicazione i salti vogliamo che i salti siano ripetuti il numero di volte necessario che permette all'atleta di prendere familiarità con essi delle varie mosse al punto che vengano eseguite con fluidità tale che si possa passare a scavalcare un secondo eventuale ostacolo con il minimo preavviso possibile.\\
\\
Per il livello di analisi gli atleti vengono messi di fronte ad un percorso che dovranno completare usando tutte e tre queste tecniche, mentre per il livello di sintesi verrano chiesti di raggiungere dal punto A un punto B, senza aver mostrato loro un percorso in particolare.
Il livello di valutazione non è necessario che venga raggiunto per lo scopo degli allenamenti.

\subsection{Modalità di apprendimento}

Il corso prevede allenamenti della durata di due ore per due volte alla settimana. La maggior parte degli allenamenti avvengono in una palestra al chiuso con ostacoli costruiti appositamente, ma quando le condizioni lo permettono e gli atleti dimostrano di essere pronti e responsabili avvengono anche all'aperto.\\
\\
Come suggerisce il Cono dell'apprendimento di Dale useremo più canali di comunicazione per le lezioni. Ognuna delle tre tecniche verranno a voce descritte e spiegate a cosa servono, dove possono essere utilizzate e quali vantaggi hanno rispetto ad altre tecniche che potrebbero essere utilizzate per scavalcare lo stesso ostacolo. Successivamente verrà mostrata la tecnica e indicati quali sono i dettagli a cui stare attenti. Infine verrà fatta provare, solitamente per un periodo tra i 10 e 20 minuti prima che venga introdotta una variazione nell'esercizio.

 \subsection{Gli atleti}

Durante le lezioni sono riuscito ad individuare 4 tipi di atleti che hanno preso parte alle lezioni.
 \begin{itemize}
  \item Alberto ha 18 anni ed è uno degli studenti più grandi in termini di età del corso. Pratica parkour da molti anni e ormai ha conoscenza di tutte le tecniche e per la maggior parte riesce anche ad eseguirle. Le poche tecniche che non riesce a fare sono i salti acrobatici più complicati e le tecniche che sono state esagerate per renderle più complicate da realizzare, come il triplo monkey o i doppi backflip. Alberto ha un buon fisico che gli permette di affrontare allenamenti fisici molto intensi. Partecipa ai corsi perchè è appassionato e per non perdere la sua forma fisica. Durante una lezione gli piace sfidare i suoi amici ad una variazione più difficile delle tecniche, allungando un salto o rendendolo più veloce o fluido. Nonostante questo è molto prudente e non corre rischi innecessari.

  \item Beatrice ha 15 anni ed molti studenti del corso sono come Beatrice. Pratica parkour da 1 o 2 anni e benche riesce a concludere la maggior parte degli allenamenti fisici con più o meno difficoltà ha ancora molta pratica da fare per perfezionare l'esecuzione delle sue tecniche. Passa la maggior parte del tempo a lezione a fare gli esercizi previsti solo occasionalmente si distrae. Beatrice non sembra particolarmente appassionata dal parkour lei per lei è uno sport come un'altro che si fa da piccoli come c'è gente che pratica tennis o pallavolo, lei pratica parkour.

  \item Carlo ha 8 anni e partecipa solo alla prima metà di lezione con il gruppo bambini. È la sua prima esperienza con parkour sebbene non pochi bambini come Carlo hanno continuato a particare la disciplina per molti anni, diventando come Alberto. Carlo ha molte energie e si distrae facilemente, spesso distraendo i suoi amici. Gli piace saltare correre e mouversi qua e la piuttosto che stare fermi a fare i classici esercizi fisici come squat e piegamenti.

  \item Dario ha 14 anni pratica parkour da 2 anni e non è molto motivato ad allenarsi. A Dario piace molto darsi delle arie anche se spesso inventa scuse per evitare un esercizio. Riesce nella maggior parte degli esercizi ma molti non li fa e quando li fa non vede l'ora che finiscano.
\end {itemize}

\subsection{L'allenamento}

Le lezioni sono della durata di due ore e dopo 10 minuti di riscaldamento gli atleti vengono divisi in due gruppi, ragazzi e bambini. I bambini partecipano solo a metà lezione nella quale faranno piccoli percorsini dove scavalchi due o tre ostacoli. I ragazzi partecipano ad entrambe le ore di allenamento nella quale gli allenamenti sono di solito divisi in un'ora di potenziamento fisico e un'ora di tecnica.

\subsection{Comportamento desiderato}

L'obbiettivo più grande del corso di parkour, oltre ad insegnare agli atleti la disciplina, è che sia divertente. Se gli studenti si divertono si ripresenteranno alle lezioni così possono continuare ad imparare lo sport. Quindi vogliamo che i partecipanti non si annoino alle lezioni, perciò quindi che non si distraggano o che non restino fermi a non fare nulla. Inoltre vogliamo che imparino a stare in sicurezza così se vogliono possono fare pratica a casa senza che si facciano male. Un'altro comportamento desiderato è che si impegnino negli esercizi di potenziamento, così aumentano maggiormente la loro forza che servirà ad eseguire meglio le varie tecniche.

\subsection{Comportamenti tipici}

Durante le lezioni ho potuto osservare i seguenti comportamenti da parte dei partecipanti:

\begin{itemize}
  \item Alberto per la maggior parte del tempo eseguiva le istruzioni senza battere ciglio. Gli esercizi di potenziamento fisico venivano fatti con ottima forma. Gli esercizi tecnici anche. Il problema è che è visibilmente annoiato, certe volte si staccava dal gruppo per due o tre minuti alla volta per tentare qualche tecnica o esercizio che veniva proposto poche volte perchè leggermente sopra il livello di capacità media del corso.

  \item Beatrice anche lei non perdeva tempo ad eseguire gli esercizi proposti tuttavia proponeva spesso di ripetere un esercizio o tecnica in particolare. Capitava che ogni tanto perdesse la motivazione e che le tecniche che faceva fossero eseguite in maniera approssimativa.

  \item Carlo non riusciva a stare fermo. La sua attenzione era breve e la sua impazienza e voglia di muoversi facevano si che gli esercizi sopratutto quelli mirati al potenziamento fisico fossero fatti il più velocemente possibile senza riguardo per la correttezza della forma.
\end{itemize}

\subsection{Soluzioni e risposta ad esse}

Per rendere tutti attenti e motivati nel rispetto delle lezioni sono stati prese le seguenti misure: \\
\\
Tutti gli esercizi sono stati ideati con piccole variazioni utili ad aumentare il livello di difficoltà. Gli esercizi di potenziamento è facile renderli più difficili ci sono un svariate variazioni di molti esercizi proprio per questo scopo. Ad esempio i piegamenti venivano eseguiti a ginocchia piegate, normalmente, a diamante e così via. Altri esercizi come gli squat venivano fatti tenendo pesi in mano. \\
Gli esercizi tecnici invece si cercava di renderli più precisi. Nel caso della tecnica del monkey, per chi veniva facile l'esercizio normalmente, veniva chiesto di atterrare in un piccolo spazio indicato, tipicamente il bordo di una superfice leggermente rialzata rispetto il pavimento. Per lo step invece, data la velocità che caraterizza la tecnica, si cercava di essere più fluidi possibili. Ovvero si cercava di non perdere velocità nello scavalcamento e di mantenere la testa alla stessa altezza. Per la roll invece di essere fatta per terra veniva fatta su piccoli travetti.\\
Alberto ha risposto bene a queste modifiche in quanto era sempre possibile aumentare la difficoltà degli esercizi in misura tale che siano stimolanti anche per lui.\\
\\
Negli esercizi di potenziamento venivano spessi introdotte sfide di resistenza. Ci sono molti esercizi che prevedono di restare fermi in una posizione come il plank o la sedia a muro. In questo modo si è introdotta un po' di competizione tra gli atleti che li motiva a dare il meglio per essere i più bravi. Questa meccanica è piaciuta molto ad Alberto e Dario.\\
\\
Per venire incontro al bisogno di Carlo di consumare le sue energie Sono stati inventati delle variazioni a giochi che la maggior parte degli atleti conoscevano già. Un esempio di questi è "Lupo gelo". In questo gioco due persone fanno da lupo e il loro compito è toccare gli altri partecipanti congelandoli sul posto. Chi veniva toccato doveva mettersi a terra in posizione di plank finchè qualcun'altro non lo saltava con un salto a piedi pari. Ogni partita durava circa dai due a tre minuti. \\
Un'altro esempio il gioco di acchiappa la coda dove un gruppo di persone con una pezza appesa alla vita cercano di rubarsi a vicenda questa "coda". La variazione introdotta fu quella che il tutto veniva fatto da accovacciati con il divieto di toccare per terra con le mani.\\
Questo tipo di esercizi hanno lo scopo di mascherare un esercizio fisico in un gioco.\\
Questo fu molto apprezzati da Dario, Carlo e Beatrice.\\

\subsection{conclusioni}

I risultati dell'introduzioni di queste piccole meccaniche di Gamificazione sono state sorprendenti vederle. Le prime lezioni cui ho svolto ero triste a vedere che i miei studenti spesso a distrarsi. In particolare gli atleti, dopo questi accorgimenti, alcuni degli studenti venivano a chiedermi particolari esercizi addiritura sfidandomi.
\newpage
